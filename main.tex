\documentclass{article}
\usepackage{hyperref}
\usepackage{fancyhdr}
\usepackage{array}
\usepackage{amssymb}
\usepackage{longtable}
\usepackage{iitem}
\usepackage{enumitem}

\hypersetup{colorlinks=true,linkcolor=black}

\makeatletter
\newcommand{\rmnum}[1]{\romannumeral #1}
\newcommand{\Rmnum}[1]{\expandafter\@slowromancap\romannumeral #1@}
\makeatother

\usepackage[top=2cm, bottom=2cm, left=1.5cm, right=1.5cm]{geometry}
% 页眉和页脚设置
\pagestyle{fancy}
\fancyhf{} % 清空默认的页眉和页脚
\fancyfoot[C]{\thepage} % 页脚中央显示页码
\renewcommand{\headrulewidth}{0.4pt} % 页眉下方的横线
\fancyhead[R]{\nouppercase{\leftmark}} % 将课程类型放在页眉右侧


% general info ------------------------------------------------------------+
\title{Course Description}
\author{--}
\date{\today}
\newcommand{\majorname}{Computer Science and Technology}
\newcommand{\collegename}{College of Computer Science}
\newcommand{\universityname}{-- University}

\begin{document}

% cover ------------------------------------------------------------+
\begin{titlepage}
 \centering
 \vspace*{4cm}
 \Huge \textbf{Course Description} \\
 \vspace{10cm}
 \begin{flushright}
  \emph{\Large\majorname}\\
  \vspace{0.1cm}
  \emph{\Large\collegename}\\
  \vspace{0.1cm}
  \emph{\Large\universityname}
 \end{flushright}
\end{titlepage}

% contents ------------------------------------------------------------+
\clearpage
\setcounter{page}{1}
\pagenumbering{roman}
\tableofcontents % 添加目录
\clearpage

\pagestyle{fancy} % 重新激活页眉和页脚
\pagenumbering{arabic}

% core course ------------------------------------------------------------+
\section{CORE COURSE}
\markright{CORE COURSE}

\subsection{High Level Language Program Design 2-1}
\renewcommand{\arraystretch}{1.5} % 设置行高
\begin{longtable}{|p{0.18\textwidth}|p{0.75\textwidth}|}

\hline% --------------------------------------------------

\textbf{Course Name} & High Level Language Program Design 2-1
\\ \hline
\textbf{Credits} & 3.5 credits = 5.25 ECTS
\\ \hline
\textbf{Credit Hours} & 80
\\ \hline
\textbf{Semester} & 2019-2020/1 
\\ \hline
\textbf{Description} & This course lays a crucial foundation for computer science and related majors, focusing on practical programming skills and computational thinking. It covers C++ fundamentals such as data types, operators, control structures, and functions, enabling students to solve engineering problems. Key topics include arrays, pointers, and the evolution of C++ standards across platforms. With project-based learning, students gain hands-on experience in software design, development methods, and modern engineering tools. By working on comprehensive projects, they will develop the skills necessary for the full software development cycle, providing a solid base for advanced studies and graduation projects. 
\\ \hline
\textbf{Education Aims} & \begin{enumerate}[topsep=0pt, partopsep=0pt, parsep=0pt, itemsep=1pt]
    \item Understand and master basic concepts in C++ such as data types, operators, and expressions, and use structured programming methods to implement complex computer engineering solutions.
    \item Grasp key concepts related to C++ control statements and functions, and identify critical aspects of complex engineering problems.
    \item Learn the differences between C++ compilation platforms, understand the evolution of C++ standards, and use advanced debugging/testing tools. Gain proficiency in C++ arrays, pointers, and the associated programming techniques.
    \item Through project-based learning, acquire the methods and technologies for full-cycle software design and development, enhancing programming and software engineering skills.
\end{enumerate}
 
\\ \hline
\textbf{Main Contents} & \textbf{Chapter 1:} Introduction to C++ Programming (3 hours) 
\begin{itemize}[topsep=0pt, partopsep=0pt, parsep=0pt, itemsep=1pt]
    \item \textbf{Topics:} Overview of programming languages, numeral systems, data storage, basic concepts of C++, structure of C++ programs. 
    \item \textbf{Key Concepts:} Binary data storage, numeral system conversions, literal constants. 
    \item \textbf{Difficulties:} Converting between numeral systems and understanding different types of literals.
\end{itemize}

\textbf{Chapter 2:} Data Types (3 hours) 
\begin{itemize}[topsep=0pt, partopsep=0pt, parsep=0pt, itemsep=1pt]
    \item \textbf{Topics:} Basic data types, compound data types, enumeration, and cv-qualified data types. 
    \item \textbf{Key Concepts:} Data representation in memory, equivalence, and type conversions. 
    \item \textbf{Difficulties:} Initialization methods, memory storage forms of different data types.
\end{itemize}

\textbf{Chapter 3:} Operators and Expressions (3 hours) 
\begin{itemize}[topsep=0pt, partopsep=0pt, parsep=0pt, itemsep=1pt]
    \item \textbf{Topics:} Assignment, arithmetic, relational, logical, bitwise, conditional operators. 
    \item \textbf{Key Concepts:} Expression evaluation. 
    \item \textbf{Difficulties:} Operator precedence and associativity.
\end{itemize}

\textbf{Chapter 4:} Control Structures, Arrays, Structures, and Unions (15 hours) 
\begin{itemize}[topsep=0pt, partopsep=0pt, parsep=0pt, itemsep=1pt]
    \item \textbf{Topics:} Branch, loop, and jump statements, arrays, structures, and unions. 
    \item \textbf{Key Concepts:} Programming with control structures and arrays. 
    \item \textbf{Difficulties:} Nested loops, array-pointer relationships.
\end{itemize}
\\
\hline% ------------------------------------------------
\textbf{} & 

\textbf{Chapter 5:} Functions (12 hours) 
\begin{itemize}[topsep=0pt, partopsep=0pt, parsep=0pt, itemsep=1pt]
    \item \textbf{Topics:} Function declarations, recursion, operator overloading. 
    \item \textbf{Key Concepts:} Function calls, recursion, parameter passing, scope, and lifetime of variables. 
    \item \textbf{Difficulties:} Recursive function calls and static variables.
\end{itemize}
    
\textbf{Chapter 6:} Pointers, References, and Dynamic Memory Allocation (12 hours) 
\begin{itemize}[topsep=0pt, partopsep=0pt, parsep=0pt, itemsep=1pt]
    \item \textbf{Topics:} Pointers, dynamic memory, linked lists, references. 
    \item \textbf{Key Concepts:} Dynamic memory allocation, pointer arithmetic. 
    \item \textbf{Difficulties:} Implementing and applying linked lists.
\end{itemize}

\textbf{Laboratory Sessions:} 
\begin{itemize}[topsep=0pt, partopsep=0pt, parsep=0pt, itemsep=1pt]
    \item Practical exercises including C++ setup, basic data types, operators, control structures, arrays, functions, and pointers. Students will use online evaluation systems to complete and submit their assignments.
\end{itemize}
 
\\ \hline
\textbf{Evaluation} & The course assessment is a blend of continuous assessment and a final exam, with the following breakdown:

\begin{itemize}[topsep=0pt, partopsep=0pt, parsep=0pt, itemsep=1pt]
    \item \textbf{Self-study assessment (5\%):} Based on video completion and pre-class quizzes, supporting course objective 2.
    \item \textbf{In-class quizzes (5\%):} Evaluated based on the accuracy of in-class quizzes, supporting course objective 2.
    \item \textbf{Lab work assessment (10\%):} Evaluated through the completion of programming assignments, supporting course objectives 2 and 3 (each contributing 50\%).
    \item \textbf{Comprehensive project development (10\%):} Assessment based on the completion of a project, supporting course objective 4.
    \item \textbf{Comprehensive lab skills (20\%):} Assessed via an on-computer exam, supporting course objectives 1, 2, and 3 (contributing 15\%, 35\%, and 50\%, respectively).
    \item \textbf{Final Exam (50\%):} A closed-book exam covering various C++ topics, supporting course objectives 1, 2, 3, and 4. It covers multiple-choice questions, code correction, code analysis, and programming tasks, each designed to reinforce specific course objectives.
\end{itemize} 
\\ \hline
\textbf{Study Materials} & \begin{itemize}[topsep=0pt, partopsep=0pt, parsep=0pt, itemsep=1pt]
    \item \textit{C++ Programming (2nd Edition)}, by Liu Jing, Higher Education Press, ISBN 9787040354560, 2013
    \item \textit{The C++ Programming Language}, by Bjarne Stroustrup, Mechanical Industry Press, ISBN 9787111539414, 2016
    \item \textit{C++ Primer (Chinese Edition)}, by S.B. Lippman and J. Lajoie, Electronics Industry Press, ISBN 9787121155352, 2014
\end{itemize} 
\\ \hline
\textbf{Language} & Chinese 
\\ \hline
\textbf{Final Grade} & 4.0
\\ \hline


\end{longtable}
% \input{courses/02_OS.tex}

\clearpage

% non-major course ------------------------------------------------------------+
\section{NON-MAJOR COURSE}
\markright{NON-MAJOR COURSE}

% a non-major course ------------------------------------------------------------+
\subsection{Course B}
\renewcommand{\arraystretch}{1.5} % 设置行高
\begin{longtable}{|p{3cm}|p{10cm}|}

\hline% --------------------------------------------------
\textbf{Course Name} & 
Course B
\\\hline% ------------------------------------------------
\textbf{Credits} & 
4.0 
\\\hline% ------------------------------------------------
\textbf{Semester} & 
Fall 
\\\hline% ------------------------------------------------

\end{longtable}


% end ------------------------

\clearpage

\end{document}
